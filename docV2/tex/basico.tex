% EXPOSICION DE CONCEPTOS BASICOS
\chapter{Conceptos b�sicos}

\begin{flushright}
  \begin{minipage}[t]{13cm}
    \begin{flushright}
      \begin{quote}
        \emph{
        Se debe hacer todo tan sencillo como sea posible, pero no m�s sencillo.
        }
        \begin{flushright}
          \textbf{\textemdash Albert Einstein}
        \end{flushright}
      \end{quote}
    \end{flushright}
  \end{minipage}
\end{flushright}

\bigskip

\begin{flushright}
  \begin{minipage}[t]{13cm}
    \begin{flushright}
      \begin{quote}
        \emph{
        Fundamental progress has to do with the reinterpretation of basic ideas.
        }
        \begin{flushright}
          \textbf{\textemdash Alfred North Whitehead}
        \end{flushright}
      \end{quote}
    \end{flushright}
  \end{minipage}
\end{flushright}

\bigskip

\begin{center}{\line(1,0){325}}\end{center}

%--------------------------------------------------------%

  
\section{Estructura general de la librer�a}\label{estructuraGeneralDeLaLiberia}
  
  \subsection{Los procesadores virtuales}

  \subsubsection{Las CPU b�sicas} 

  \paragraph{La CPU escalar}

  \paragraph{La CPU SIMD}\label{basico:cpusimd}

  \subsubsection{La CPU vectorial} se encarga de las operaciones sobre los
  vectores (polinomios) que se utilizan para representar los n�meros,
  como se expone en \ref{representacionZ}. Descansa totalmente
  sobre la capa inmediatamente inferior, la CPU b�sica.
  Implementa tan s�lo operaciones en $\mathbb{Z}_0$: el
  tratamiento del signo vendr� en capas superiores. Si la CPU b�sica
  era importante, esta capa lo es mucho m�s: el grueso del manejo de
  operaciones en m�ltiple precisi�n se encuentra aqu�.

  \subsection{Funciones}\label{funcionesBasico}
    
  \subsection{Control de errores}\label{controlDeErrores}


