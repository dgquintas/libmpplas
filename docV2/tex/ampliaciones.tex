% CAPITULO SOBRE LAS POSIBLES MEJORAS, AMPLIACIONES Y CRITICA

\chapter{Cr�tica}

\begin{flushright}
  \begin{minipage}[t]{13cm}
    \begin{flushright}
      \begin{quote}
        \emph{
          Become addicted to constant and never-ending self improvement.
        }
        \begin{flushright}
          \textbf{\textemdash Anthony J. D'Angelo, The College Blue Book}
        \end{flushright}
      \end{quote}
    \end{flushright}
  \end{minipage}
\end{flushright}

\begin{center}{\line(1,0){325}}\end{center}

%--------------------------------------------------------%

\section{Posibles mejoras}
  \subsection{Mayor aprovechamiento de instrucciones SIMD}
    En las operaciones con matrices.

\section{Ampliaciones}
  \subsection{Sobre Polinomios}
    \subsubsection{Factorizaci�n}
      Al igual que para la factorizaci�n de enteros, no se conoce
      un m�todo que permita realizar esta operaci�n en un tiempo razonable. Como es habitual
      en este contexto, �razonable� suele traducirse como �computable en tiempo polinomial�.
      %TODO
  \subsection{Sobre cuerpos finitos}
    En la implementaci�n actual de los cuerpos finitos $\GF{(p^n)}$, 
    se modela $p$ como un entero de tipo \texttt{mpplas::Z}; es decir,
    un entero de precisi�n arbitraria. En la mayor�a de los casos, la caracter�stica de la precisi�n arbitraria
    no es explotada, y de hecho influye negativamente en el rendimiento. De hecho, es usual el uso de cuerpos
    finitos de la forma $\GF{(2^n)}$. En este caso, es posible explotar el car�cter binario de los elementos.
    Por ejemplo, la suma se reduce a una operaci�n o-exclusivo a nivel de bits. 

    En cualquier caso, la presente biblioteca se centra entorno a tipos de precisi�n arbitraria, raz�n por la cual
    esta caracter�stica tendr�a m�s cabida en otro tipo de biblioteca.


\section{Conclusiones}
  

