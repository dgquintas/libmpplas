% PRUEBAS BORRADOR

\chapter{Pruebas, \textbf{BORRADOR}}
\section{Casos de prueba por funci�n}
\begin{itemize}
  \item divMP(vector, vector)
    \begin{itemize}
      \item[-] Ejemplo: Un entero igual a
        $79228162514264337593543950336 = 2^{32+64}$ y otro igual a 
        $4294967296 = 2^{32}$. Su divisi�n provoca a fecha 21 de Enero 
        que falle la divisi�n debido a que el tama�o del dividendo 
        disminuye m�s r�pido que de uno en uno, mientras que la guarda
        del bucle ``principal'' disminuye de uno en uno.\\
        \textbf{Soluci�n}: La guarda del bucle debe actualizarse en
        funci�n de la variaci�n de tama�o del dividendo.
    \end{itemize}
\end{itemize}

        
        
