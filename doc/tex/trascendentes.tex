% CAPITULO SOBRE FUNCIONES TRASCENDENTES

\chapter{Funciones trascendentes}\label{cap:primos}

\begin{flushright}
  \begin{minipage}[t]{13cm}
    \begin{flushright}
      \begin{quote}
        \emph{
         cita
        }
        \begin{flushright}
          \textbf{\textemdash D. Zagier}
        \end{flushright}
      \end{quote}
    \end{flushright}
  \end{minipage}
\end{flushright}

\bigskip

\begin{flushright}
  \begin{minipage}[t]{13cm}
    \begin{flushright}
      \begin{quote}
        \emph{
        cita 2
        }
        \begin{flushright}
          \textbf{\textemdash H. Montgomery}
        \end{flushright}
      \end{quote}
    \end{flushright}
  \end{minipage}
\end{flushright}

\bigskip

\begin{center}{\line(1,0){325}}\end{center}

%--------------------------------------------------------%

\section{Introducci�n}
  A function which is not an algebraic function.
  In other words, a function which "transcends," i.e., cannot be
  expressed in terms of, algebra. Examples of transcendental functions
  include the exponential function, the trigonometric functions, and the
  inverses functions of both.

\section{La funci�n exponencial}
  
  \subsection{Exponencial inversa: la funci�n logaritmo}
    \begin{observacion}\label{obsLog}
      Si $x \in \mathbb{R}$, existe un $n \in \mathbb{N}$ tal que
      $2^{n-1} < x \leq 2^n$. Entonces, si $y = x/2^n$, se verifica
      que $0 < y \leq 1$
    \end{observacion}

    En base a la observaci�n \ref{obsLog}, el c�lculo de $x \in
    \mathbb{R}$ se reducir�a a (utilizando los s�mbolos de dicha
    observaci�n):
    \[
      \log{(x)} = \log{(y \times 2^n)} = \log{(y)} + n \cdot \log{(2)}
    \]
    Las ventajas que de esta forma de calcula el logaritmo se
    desprenden son claras: por una parte, el c�lculo de $\log(2)$ (tambi�n denominada
    ``La constante logar�tmica'', v�ase \cite{log2}) es un problema
    ampliamente tratado y existen m�todos de gran eficiencia para su c�lculo. 
    M�s adelante se tratar� este punto en m�s profundidad.
    Por otra parte, al cumplirse $0 < y \leq 1$, el c�lculo de
    $\log{(y)}$ puede realizarse satisfactoriamente y con relativa
    efectividad utilizando la expansi�n de MacLaurin
    para la funci�n logaritmo.

    \subsubsection{La constante logar�tmica}

    


  
\section{Funciones trigonom�tricas}

  \subsection{El c�lculo de las inversas}


