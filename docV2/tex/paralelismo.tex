% CAPITULO SOBRE LA IMPLEMENTACION DE XXX

\chapter{ XXX }\label{cap:XXX}

\begin{flushright}
  \begin{minipage}[t]{13cm}
    \begin{flushright}
      \begin{quote}
        \emph{

          xxxxxxxxxxxxxxxxxxxx

        }
        \begin{flushright}
          \textbf{\textemdash xxxxxxxxxxxxx}
        \end{flushright}
      \end{quote}
    \end{flushright}
  \end{minipage}
\end{flushright}

\bigskip

\begin{flushright}
  \begin{minipage}[t]{13cm}
    \begin{flushright}
      \begin{quote}
        \emph{

        xxxxxxxxxxxxxxxxxxxxxxxx

        }
        \begin{flushright}
          \textbf{\textemdash xxxxxxxxxxxxx}
        \end{flushright}
      \end{quote}
    \end{flushright}
  \end{minipage}
\end{flushright}

\bigskip

\begin{center}{\line(1,0){325}}\end{center}

%--------------------------------------------------------%

\section{Introducci�n}


Es com�n confundir los t�rminos \emph{concurrente} y \emph{paralelo},
intercambiandolos y/o utiliz�ndolos de forma incorrecta. Es posible
que esto sea debido a que ambos comparten la idea de realizar varios
trabajos ``al mismo tiempo''. Sin m�s pre�mbulo, sirva para aclarar la
diferencia las correspondientes definiciones:

\begin{definicion}[Computaci�n paralela]
\end{definicion}

\begin{definicion}[Computaci�n concurrente]
\end{definicion}

Por tanto, en toda computaci�n paralela hay impl�cito un concepto de
concurrencia -la ejecuci�n de diferentes tareas simult�neamente-, pero
hablar de concurrencia \emph{no} implica necesariamente parelelismo: las
tareas ejecutadas concurrentemente no tienen porque estar
relacionadas o corresponder a la resoluci�n de un �nico problema.


\section{Problematica}

  \subsection{Asegurando la correcci�n}
    Uno de los puntos m�s problem�ticos al cambiar del paradigma
    secuencial al concurrente

