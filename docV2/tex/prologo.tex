\chapter*{Pr�logo}

\section*{Peque�o apunte sem�ntico}
A lo largo de esta documentaci�n, se ha evitado utilizar el
\emph{relativamente} incorrecto, aunque desafortunadamente ampliamente extendido,
t�rmino �librer�a� para designar el t�rmino t�cnico ingl�s
�library�\footnote{la Real Academia de la Lengua considera ambos
t�rminos sin�nimos. Sin embargo solamente para las acepciones cl�sicas
del t�rmino �librer�a� como local en el cual se tienen libros}.
La definici�n de �library� en este contexto, seg�n el diccionario 
Webster's Revised Unabridged Dictionary (1913): 
\begin{quote}
Library: (...)
   1. A considerable collection of books kept for use, and not
      as merchandise; as, a private library; a public library.
\end{quote}
Es decir, lo que corresponder�a a la definici�n del t�rmino castellano 
�biblioteca�: lugar en el cual se almacena informaci�n lista para su
uso, a \textit{grosso modo} en nuestro caso, funciones; de forma
opuesta a un lugar donde se venden libros. El uso del t�rmino �librer�a�
tendr�a su correspondencia en �bookshop�, como lugar de \emph{venta} de
libros.

Est� claro que esta distinci�n es relativamente trivial por purista en
el uso del lenguaje. Sin embargo, no existe raz�n para no tratar de
tratar de minimizar la impureza en cualquier forma de comunicaci�n. A�n 
con m�s raz�n cuando se trata de llevar a cabo con rigor. No se podr�a
esperar menos de un trabajo que pretende ir de la mano de la estricta
disciplina Matem�tica.  
