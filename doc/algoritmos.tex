\documentclass[a4paper,twoside,spanish]{article}
\usepackage{pslatex} %utilizar fuentes postscript standard
\usepackage[T1]{fontenc} %codificacion de tipo
\usepackage[spanish]{babel}
\usepackage[latin1]{inputenc} %entrada usando ISO-Latin1
\usepackage{makeidx} %para los indices
\usepackage{amssymb}
\usepackage{amsmath}
\usepackage{euler}

\usepackage[section]{algorithm}
\usepackage{algorithmic}
\floatname{algorithm}{Algoritmo}
\renewcommand{\listalgorithmname}{Lista de algoritmos}

\usepackage{makeidx}
%\usepackage{hyperref}
%\hypersetup{colorlinks,%
%            pdftex,%
%            linkcolor=black,%
%            citecolor=black,%
%            filecolor=black,%
%            urlcolor=black,%
%						pdfpagemode={None}		}
\frenchspacing

\usepackage{type1cm}


\begin{document}
% IMPLEMENTACION DE LOS ALGORITMOS

%\section{Multiplicaci�n}
%\subsection{Caso base}
%Como se describe en \cite{lucena} 7.2.3 y \cite{gmp} 16.1.1, 

\section{Divisi�n}
Puede chocar que se requiera que $n>1$: se utiliza otro algoritmo para
la divisi�n entre una sola cifra (es este mismo algoritmo pero
simplificado, claro --- sin iteraciones.)

\begin{algorithm}
  \caption{Algoritmo cl�sico de divisi�n}
  \begin{algorithmic}
    \REQUIRE $u=(u_1u_2\ldots u_{m+n})_b$, $v=(v_1v_2\ldots v_n)_b$ no
    negativos. $v_1 \neq 0$ y $n>1$.

    \COMMENT{Normalizaci�n. Para tener $v_1 >= \lfloor b/2 \rfloor$}
    \STATE $d \leftarrow \textrm{N� de ceros a la izquierda del
    primer $1$ significativo de } v_1$
    \STATE $(u_1u_2\ldots u_{m+n})_b \leftarrow (u_1u_2\ldots u_{m+n})_b \cdot 2^d$
    \STATE $(v_1v_2\ldots v_{n})_b \leftarrow (v_1v_2\ldots v_{n})_b \cdot 2^d$\\*

    \COMMENT{Comienza el bucle}
    \FOR{ $j = 0$ hasta $j \leq m$ }
    \IF{ $u_j = v_1$}
        \STATE $\hat{q} \leftarrow b-1$
      \ELSE
        \STATE $\hat{q} \leftarrow \lfloor \frac{u_{j}b+u_{j+1}}{v_1}
              \rfloor$
      \ENDIF
      \WHILE{ $v_{2}\hat{q} > (u_{j}b+u_{j+1}-\hat{q}v_1)b + u_{j+2}$}
        \STATE $\hat{q} \leftarrow \hat{q} - 1$
      \ENDWHILE
      
      $(u_ju_{j+1}\ldots u_{j+n})_b \leftarrow (u_ju_{j+1}\ldots u_{j+n})_b - \hat{q}(v_1v_2\ldots v_{n})_b$
      \COMMENT{$(u_ju_{j+1}\ldots u_{j+n})_b$ debe mantenerse
      positivo. Si fuera negativo se le sumar�a $b^{n+1}$ con el fin de
      obtener su representacion en complemento a $b$}

      \STATE $q_j \leftarrow \hat{q}$
      \IF{$(u_ju_{j+1}\ldots u_{j+n})_b$ fue negativo}
        \STATE $q_j \leftarrow q_j -1$
        \STATE $(u_ju_{j+1}\ldots u_{j+n})_b \leftarrow (u_ju_{j+1}\ldots u_{j+n})_b + (0v_1v_2\ldots v_n)_b$
        \COMMENT{Ignorar el acarreo que se produce a la izquierda de
        $u_j$, debido al complemento en base $b$}
      \ENDIF
    \ENDFOR
    \COMMENT{Desnormalizar, para el resto}
    \STATE $(u_{m+1}u_{m+2}\ldots u_{m+n})_b \leftarrow (u_{m+1}u_{m+2}\ldots u_{m+n})_b / 2^d$
    \STATE Devolver $(q_{0}q_{1}\ldots q_{m})_b$ como cociente.
    \STATE Devolver $(u_{m+1}u_{m+2}\ldots u_{m+n})_b$ como resto.


    \ENSURE Se forma el producto $\lfloor u/v \rfloor = (q_0q_1\ldots
    q_m)_b$ y el resto $u \bmod v = (r_1r_2\ldots r_n)_b$
  \end{algorithmic}
\end{algorithm}

Se cita que la probabilidad de que el nuevo dividendo tras la resta
sea negativo es del orden de $2/b$ con $b$ la base.

Una forma de verlo es:
\begin{equation}
\textrm{Si } 
\left.
\begin{array}{l}
  v_1 \geq \lfloor b/2 \rfloor \\ 
  v_2\hat{q}\leq b\hat{r}+u_2 \\
  \hat{q} \neq q  
\end{array}\right\}
\Longrightarrow u \bmod u \geq (1-2/b)v
\end{equation}
Esto ultimo ocurre con aprox. frecuencia $2/b$, as� que cuando $b$ es
el ancho de palabra (base) utilizado, tendremos $q_j = \hat{q}$
excepto en raras ocasiones (precisamente en las que la resta para
conformar el nuevo dividendo tiene un resultado $<0$).
\end{document}
